<!DOCTYPE html>
<!-- Generated by pkgdown: do not edit by hand --><html lang="en"><head><meta http-equiv="Content-Type" content="text/html; charset=UTF-8"><meta charset="utf-8"><meta http-equiv="X-UA-Compatible" content="IE=edge"><meta name="viewport" content="width=device-width, initial-scale=1, shrink-to-fit=no"><meta name="description" content="Rticles: scientific documents with R"><title>Rticles • inti</title><!-- favicons --><link rel="icon" type="image/png" sizes="16x16" href="../favicon-16x16.png"><link rel="icon" type="image/png" sizes="32x32" href="../favicon-32x32.png"><link rel="apple-touch-icon" type="image/png" sizes="180x180" href="../apple-touch-icon.png"><link rel="apple-touch-icon" type="image/png" sizes="120x120" href="../apple-touch-icon-120x120.png"><link rel="apple-touch-icon" type="image/png" sizes="76x76" href="../apple-touch-icon-76x76.png"><link rel="apple-touch-icon" type="image/png" sizes="60x60" href="../apple-touch-icon-60x60.png"><script src="../deps/jquery-3.6.0/jquery-3.6.0.min.js"></script><meta name="viewport" content="width=device-width, initial-scale=1, shrink-to-fit=no"><link href="../deps/bootstrap-5.3.1/bootstrap.min.css" rel="stylesheet"><script src="../deps/bootstrap-5.3.1/bootstrap.bundle.min.js"></script><!-- Font Awesome icons --><link rel="stylesheet" href="https://cdnjs.cloudflare.com/ajax/libs/font-awesome/5.12.1/css/all.min.css" integrity="sha256-mmgLkCYLUQbXn0B1SRqzHar6dCnv9oZFPEC1g1cwlkk=" crossorigin="anonymous"><link rel="stylesheet" href="https://cdnjs.cloudflare.com/ajax/libs/font-awesome/5.12.1/css/v4-shims.min.css" integrity="sha256-wZjR52fzng1pJHwx4aV2AO3yyTOXrcDW7jBpJtTwVxw=" crossorigin="anonymous"><!-- bootstrap-toc --><script src="https://cdn.jsdelivr.net/gh/afeld/bootstrap-toc@v1.0.1/dist/bootstrap-toc.min.js" integrity="sha256-4veVQbu7//Lk5TSmc7YV48MxtMy98e26cf5MrgZYnwo=" crossorigin="anonymous"></script><!-- headroom.js --><script src="https://cdnjs.cloudflare.com/ajax/libs/headroom/0.11.0/headroom.min.js" integrity="sha256-AsUX4SJE1+yuDu5+mAVzJbuYNPHj/WroHuZ8Ir/CkE0=" crossorigin="anonymous"></script><script src="https://cdnjs.cloudflare.com/ajax/libs/headroom/0.11.0/jQuery.headroom.min.js" integrity="sha256-ZX/yNShbjqsohH1k95liqY9Gd8uOiE1S4vZc+9KQ1K4=" crossorigin="anonymous"></script><!-- clipboard.js --><script src="https://cdnjs.cloudflare.com/ajax/libs/clipboard.js/2.0.11/clipboard.min.js" integrity="sha512-7O5pXpc0oCRrxk8RUfDYFgn0nO1t+jLuIOQdOMRp4APB7uZ4vSjspzp5y6YDtDs4VzUSTbWzBFZ/LKJhnyFOKw==" crossorigin="anonymous" referrerpolicy="no-referrer"></script><!-- search --><script src="https://cdnjs.cloudflare.com/ajax/libs/fuse.js/6.4.6/fuse.js" integrity="sha512-zv6Ywkjyktsohkbp9bb45V6tEMoWhzFzXis+LrMehmJZZSys19Yxf1dopHx7WzIKxr5tK2dVcYmaCk2uqdjF4A==" crossorigin="anonymous"></script><script src="https://cdnjs.cloudflare.com/ajax/libs/autocomplete.js/0.38.0/autocomplete.jquery.min.js" integrity="sha512-GU9ayf+66Xx2TmpxqJpliWbT5PiGYxpaG8rfnBEk1LL8l1KGkRShhngwdXK1UgqhAzWpZHSiYPc09/NwDQIGyg==" crossorigin="anonymous"></script><script src="https://cdnjs.cloudflare.com/ajax/libs/mark.js/8.11.1/mark.min.js" integrity="sha512-5CYOlHXGh6QpOFA/TeTylKLWfB3ftPsde7AnmhuitiTX4K5SqCLBeKro6sPS8ilsz1Q4NRx3v8Ko2IBiszzdww==" crossorigin="anonymous"></script><!-- pkgdown --><script src="../pkgdown.js"></script><meta property="og:title" content="Rticles"><meta property="og:description" content="Rticles: scientific documents with R"><meta property="og:image" content="https://inkaverse.com/reference/figures/logo.png"><!-- mathjax --><script src="https://cdnjs.cloudflare.com/ajax/libs/mathjax/2.7.5/MathJax.js" integrity="sha256-nvJJv9wWKEm88qvoQl9ekL2J+k/RWIsaSScxxlsrv8k=" crossorigin="anonymous"></script><script src="https://cdnjs.cloudflare.com/ajax/libs/mathjax/2.7.5/config/TeX-AMS-MML_HTMLorMML.js" integrity="sha256-84DKXVJXs0/F8OTMzX4UR909+jtl4G7SPypPavF+GfA=" crossorigin="anonymous"></script><!--[if lt IE 9]>
<script src="https://oss.maxcdn.com/html5shiv/3.7.3/html5shiv.min.js"></script>
<script src="https://oss.maxcdn.com/respond/1.4.2/respond.min.js"></script>
<![endif]--><!-- Global site tag (gtag.js) - Google Analytics --><script async src="https://www.googletagmanager.com/gtag/js?id=G-804H2EJ8RN"></script><script>
  window.dataLayer = window.dataLayer || [];
  function gtag(){dataLayer.push(arguments);}
  gtag('js', new Date());

  gtag('config', 'G-804H2EJ8RN');
</script></head><body>
    <a href="#main" class="visually-hidden-focusable">Skip to contents</a>
    

    <nav class="navbar fixed-top navbar-light navbar-expand-lg bg-dark" data-bs-theme="light"><div class="container">
    
    <a class="navbar-brand me-2" href="../index.html">inti</a>

    <small class="nav-text text-muted me-auto" data-bs-toggle="tooltip" data-bs-placement="bottom" title="">0.6.5</small>

    
    <button class="navbar-toggler" type="button" data-bs-toggle="collapse" data-bs-target="#navbar" aria-controls="navbar" aria-expanded="false" aria-label="Toggle navigation">
      <span class="navbar-toggler-icon"></span>
    </button>

    <div id="navbar" class="collapse navbar-collapse ms-3">
      <ul class="navbar-nav me-auto"><li class="nav-item">
  <a class="nav-link" href="../index.html">
    <span class="fa fa-home fa-lg"></span>
     
  </a>
</li>
<li class="nav-item">
  <a class="nav-link" href="../articles/apps.html">
    <span class="fa fa-gamepad fa-lg"></span>
     
    Apps
  </a>
</li>
<li class="active nav-item dropdown">
  <a href="#" class="nav-link dropdown-toggle" data-bs-toggle="dropdown" role="button" aria-expanded="false" aria-haspopup="true" id="dropdown--articles">
    <span class="fa fa-book fa-lg"></span>
     
    Articles
  </a>
  <div class="dropdown-menu" aria-labelledby="dropdown--articles">
    <a class="dropdown-item" href="../articles/tarpuy.html">Tarpuy: field-book experimental plans</a>
    <a class="dropdown-item" href="../articles/yupana.html">Yupana: experimental design analysis</a>
    <a class="dropdown-item" href="../articles/extra/yupana-coding.html">Yupana: coding workflow</a>
    <a class="dropdown-item" href="../articles/rticles.html">Rticles: scientific documents with R</a>
    <div class="dropdown-divider"></div>
    <a class="dropdown-item" href="../articles/heritability.html">Broad-sense heritability in plant breeding</a>
  </div>
</li>
<li class="nav-item">
  <a class="nav-link" href="../articles/policy.html">
    <span class="fa fa-shield-alt fa-lg"></span>
     
    Privacy Policy
  </a>
</li>
<li class="nav-item">
  <a class="nav-link" href="../reference/index.html">
    <span class="fa fa-file-code fa-lg"></span>
     
    Functions
  </a>
</li>
<li class="nav-item">
  <a class="nav-link" href="../news/index.html">
    <span class="fa fa-file-alt fa-lg"></span>
     
    News
  </a>
</li>
      </ul><form class="form-inline my-2 my-lg-0" role="search">
        <input type="search" class="form-control me-sm-2" aria-label="Toggle navigation" name="search-input" data-search-index="../search.json" id="search-input" placeholder="Search for" autocomplete="off"></form>

      <ul class="navbar-nav"><li class="nav-item">
  <a class="external-link nav-link" href="https://github.com/flavjack/inti/">
    <span class="fab fa fab fa-github fa-lg"></span>
     
  </a>
</li>
<li class="nav-item">
  <a class="external-link nav-link" href="https://github.com/sponsors/Flavjack">
    <span class="fas fa fas fa-heart fa-lg"></span>
     
  </a>
</li>
      </ul></div>

    
  </div>
</nav><div class="container template-article">


<div class="row">
  <main id="main" class="col-md-9"><div class="page-header">
      <img src="../logo.png" class="logo" alt=""><h1>Rticles</h1>
                        <h4 data-toc-skip class="author">Flavio Lozano-Isla</h4>
            
      
      <small class="dont-index">Source: <a href="https://github.com/flavjack/inti/blob/HEAD/vignettes/rticles.Rmd" class="external-link"><code>vignettes/rticles.Rmd</code></a></small>
      <div class="d-none name"><code>rticles.Rmd</code></div>
    </div>

    
    
El paquete \texttt{inti} permite usar un plantilla (template) para la generar documentos técnico/científicos (i.e.~tesis y artículos) usando \href{https://quarto.org/}{\texttt{Quarto}}.

\section{Herramientas}\label{herramientas}

Para el desarrollo de documentos técnico/científicos con R, deben crearse algunas cuentas e instalar los programas que necesitamos. La mayoria de estas herramientas son libres e independientes del sistema operativo y pueden ser usadas para investigación reproducible.

\begin{quote}
La lista de herramientas es una recomendación basada en mi experiencia, y no son las únicas disponibles.
\end{quote}

\subsection{Cuentas}\label{cuentas}

\begin{quote}
Se recomienda usar el mismo correo para todas las cuentas. El uso de correos diferentes para cada servicio dificultará el flujo de trabajo posteriormente.
\end{quote}

Deben crearse una cuenta en los siguientes servicios:

\begin{enumerate}
\def\labelenumi{\arabic{enumi}.}
\item
  \texttt{Google\ (Gmail)}. Se recomienda que tengan una cuenta de Google ya que nos permitirá tener acceso a \texttt{Google\ Suit} que posee un conjunto de herramientas gratuitas en línea. Estas herramientas son un buen complemento para el trabajo en equipo y puedes acceder a ellos desde distintos dispositivos móviles.
\item
  \texttt{Zotero}. Será nuestra biblioteca virtual, y una de las herramientas que más usaremos, ya que nos permitirá organizar nuestro trabajo y citar los documentos en nuestros documentos
\item
  \texttt{GitHub} (opcional). Es un servicio de repositorio de código. Nos ayudará organizar nuestros proyectos y códigos. Nos permite visualizar los historiales de cambio de nuestro proyecto, compartir nuestro código y la posibilidad de generar páginas webs.
\end{enumerate}

\subsubsection{Links para crear las cuentas}\label{links-para-crear-las-cuentas}

\subsection{Programas}\label{programas}

\begin{quote}
Instalar los programas en el orden que se mencionan para evitar conflictos en su funcionamiento.
\end{quote}

\begin{enumerate}
\def\labelenumi{\arabic{enumi}.}
\item
  \texttt{Zotero}. Es un gestor de referencias bibliográficas, libre, abierto y gratuito desarrollado por el Center for History and New Media de la Universidad George Mason.
\item
  \texttt{R\ CRAN}. Es un entorno de lenguaje de programación con un enfoque al análisis estadístico. El software R viene por defecto con funcionalidades básicas y para ampliar estas debemos instalar paquetes. R actualmente nos permite hacer distintas tareas comó análisis estadísticos, generación de gráficos, escritura de documentos, desarrollo de aplicaciones webs, etc.
\item
  \texttt{RStudio}. RStudio es un entorno de desarrollo integrado para el lenguaje de programación R, dedicado a la computación estadística y gráficos.
\item
  \texttt{Git}. Git es un software de control de versiones. Esta pensando en la eficiencia y la confiabilidad del mantenimiento de versiones de aplicaciones. Git tambien nos permitirá usar \texttt{bash} en windows a través del terminal en RStudio.
\end{enumerate}

\subsubsection{Links de los programas para instalar}\label{links-de-los-programas-para-instalar}

\paragraph{Complementos para zotero}\label{complementos-para-zotero}

\subsection{Extras}\label{extras}

Existen alguna herramientas básicas que NO deben faltar en tú computador:

\begin{itemize}
\tightlist
\item
  Chrome (buscador web)
\item
  Foxit Reader (lector de PDFs)
\item
  WinRAR (compression/descompresor de archivos)
\item
  Google Backup and Sync (servicio de sincronización de datos)
\item
  ShareX (herramienta para captura de pantalla)
\end{itemize}

Los usuarios de \texttt{Windows}, pueden instalar estas aplicaciones entre otras desde \texttt{ninite}.

\subsection{Chocolatey (opcional)}\label{chocolatey-opcional}

Si eres usuario de windows, puedes instalar todas las herramientas mencionadas desde el administrador de paquetes \texttt{chocolatey} a través de \texttt{PowerShell}.

\begin{Shaded}
\begin{Highlighting}[]
\NormalTok{open https}\SpecialCharTok{:}\ErrorTok{//}\NormalTok{chocolatey.org}\SpecialCharTok{/}\NormalTok{packages}

\NormalTok{Start}\SpecialCharTok{{-}}\NormalTok{Process powershell }\SpecialCharTok{{-}}\NormalTok{Verb runAs}

\NormalTok{Set}\SpecialCharTok{{-}}\NormalTok{ExecutionPolicy Bypass }\SpecialCharTok{{-}}\NormalTok{Scope Process }\SpecialCharTok{{-}}\NormalTok{Force; [System.Net.ServicePointManager]}\SpecialCharTok{::}\NormalTok{SecurityProtocol }\OtherTok{=}\NormalTok{ [System.Net.ServicePointManager]}\SpecialCharTok{::}\NormalTok{SecurityProtocol }\SpecialCharTok{{-}}\NormalTok{bor }\DecValTok{3072}\NormalTok{; }\FunctionTok{iex}\NormalTok{ ((New}\SpecialCharTok{{-}}\NormalTok{Object System.Net.WebClient)}\FunctionTok{.DownloadString}\NormalTok{(}\StringTok{\textquotesingle{}https://chocolatey.org/install.ps1\textquotesingle{}}\NormalTok{))}

\NormalTok{choco install googlechrome}
\NormalTok{choco install winrar}
\NormalTok{choco install zotero}
\NormalTok{choco install r}
\NormalTok{choco install rtools}
\NormalTok{choco install r.studio}
\NormalTok{choco install git}
\NormalTok{choco install google}\SpecialCharTok{{-}}\NormalTok{backup}\SpecialCharTok{{-}}\NormalTok{and}\SpecialCharTok{{-}}\NormalTok{sync}
\NormalTok{choco install foxitreader}
\NormalTok{choco install sharex}
\end{Highlighting}
\end{Shaded}

  </main><aside class="col-md-3"><nav id="toc"><h2>On this page</h2>
    </nav></aside></div>



    <footer><div class="pkgdown-footer-left">
  <p>Desarrollado por <a href="https://lozanoisla.com/" class="external-link">Flavio Lozano-Isla</a>.</p>
</div>

<div class="pkgdown-footer-right">
  <p>Site built with <a href="https://pkgdown.r-lib.org/" class="external-link">pkgdown</a> 2.0.9.</p>
</div>

    </footer></div>

  

  

  </body></html>
